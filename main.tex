\documentclass{article}
\usepackage{graphicx} % Required for inserting images
\usepackage{hyperref}

\title{Reprezentacija izvornog koda - Rust kompajler}
\author{Aleksa Bajat}
\date{September 2024}

\begin{document}

\maketitle
\newpage
\tableofcontents
\newpage

\section{Uvod}

U svetu razvoja sistema visokih performansi, operativnih sistema, drajvera i bilo kog drugog kritičnog softvera koriste se jezici niskog nivoa kao što su C i C++.
Tokom godina Microsoft je uvideo da je preko 70\% bezbednosnih slabosti nastalo pogrešnom manipulacijom i korupcijom memorije [1]. 
C nema odbrambene mehanizme od ovakvih problema (pored potencijalnih dodatnih upozorenja koja se mogu ostvariti dvostrukom upotrebom GCC i Clang kompajlera), dok C++ uprkos dodavanju konstrukta kao što su \verb|unique_ptr|, \verb|shared_ptr| i \verb|weak_ptr| interoperabilnost sa starijim standardima ili bibliotekama zahteva korišćenje sirovih pokazivača. 

Rust jezik je nastao sa ciljem da reši ovaj problem. Podrazumevana upotreba jezika je inherentno bezbedna sa aspekta memorije i paralelizma. Bezbednost se ostvaruje robustnim sistemom tipova, novog načina rukovanja memorijom i veoma striktnog \verb|rustc| kompajlera.
U C++ ekosistemu koriste se softverski obrasci kao što je \verb|Scope-bound Resource Management| da bi se standardizovala alokacija i dealkoacija memorije, ali korisnik jezika nije primoran da softvrski obrazac upotrebi. U Rust jeziku izbor ne postoji, na kraju svakog opsega vrši se automatska dealokacija memorije preko \verb|borrow checker| mehanizma (nije potreban manuelni poziv \verb|free|).
C i C++ poseduju veoma dvosmislene i neprecizne poruke kada se greška dogodi dok Rust jezik karakterišu veoma precizne poruke koje opisuju gde se desila greška i kako je potencijalno rešiti.

Prvobitna verzija Rust kompajlera je napisana u \verb|O'Caml| jeziku, dok je svaka sledeća verzija napisana u samom Rust-u.
Medjutim treba imati na umu da je Rust samo \verb|frontend| nad \verb|LLVM|-om. LLVM je kolekcija modularnih i ponovo iskoristivih tehnologija za izradu kompajlera. Rust se oslanja u potpunosti na \verb|LLVM| prilikom generisanja mašinskog koda, dok je svaki drugi deo
(\verb|frontend|) napisan potpuno ispočetka.
Veliki deo Rust jezika je izuzetno inovativan i samim tim to je razlog da se detaljno istraži kako u pozadini funkcioniše.

Reprezentacija izvornog koda jeste prikaz ili skup prikaza izvornog koda korisnika jezika da bi se obezbedili tipovi, dijagnostika,
automatska dealokacija memorije i ostali ergonomski i funkcionalni delovi jezika.
\newpage


\section{Verzije Rust kompajlera}

Rust kompajler u svakom trenutku poseduje tri glavne verzije skupa alata (\verb|toolchain|):

\begin{enumerate}
    \item \verb|Stable| - preporučena za svakodnevnu upotrebu. 
    \item \verb|Beta| - sa funkcionalnostima koje se usvajaju u sledećoj stabilnoj verziji (\verb|beta| postaje \verb|stable|).
    \item \verb|Nightly| - sa eksperimentalnim/nestabilnim funkcionalnostima. 
\end{enumerate}

\verb|Toolchain| se može promeniti na više načina ali najjednostavniji je:
\begin{verbatim}
    rustup toolchain install [version]
    rustup override set [version]
\end{verbatim}

Prikaz dostupnih lokalnih verzija, kao i trenutna aktivna verzija se može prikazati upotrebom (izlaz prikazan ispod naredbe):
\begin{verbatim}
    rustup show

    installed toolchains
    --------------------
    
    stable-x86_64-unknown-linux-gnu (default)
    nightly-2024-06-30-x86_64-unknown-linux-gnu
    nightly-2024-07-02-x86_64-unknown-linux-gnu
    nightly-x86_64-unknown-linux-gnu
    stage0
    stage1
    
    active toolchain
    ----------------

    nightly-x86_64-unknown-linux-gnu (directory 
    override for '/home/abajat/Documents/projects/master')
    rustc 1.83.0-nightly (12b26c13f 2024-09-07)
\end{verbatim}

Rust ima obećanje da će zauvek biti kompatibilan unazad uprkos konstantnom razvoju.

\newpage

\section{Upotreba Rust kompajlera}

Rust kompajler \verb|rustc| sa sastoji iz izuzetno velikog broja modula. 
Medjutim upotreba kompajlera je krajnje jednostavna.
Izvorni kod se prenese u tekstualni dokument i poziva se sledeća naredba:
\begin{verbatim}
    rustc filename.rs
\end{verbatim}
nakon čega se kreira izvršni fajl pod istim imenom (pod uslovom da je kod ispravan).

Napredni korisnici Rust-a upotrebom promenljivih okruženja i zastavica funkcionalnosti 
(\verb|feature flag|) uvode ili menjaju osnovno ponašanje \verb|rustc| kompajlera.
Funkcionalnosti korisne za razvoj Rust kompajlera su omogućene samo u \verb|Nightly| verziji koje se
ispoljavaju preko -Z (nestabilnih) zastavica. Mogu se prikazati upotrebom:

\begin{verbatim}
    rustc -C help # stable
    rustc -Z help # unstable
\end{verbatim}


\section{Reprezentacije izvornog koda}

Reprezentacije izvornog koda Rust-a se dele na:

\begin{enumerate}    
    \item Tok tokena
    \item Apstraktno sintaksno stablo
    \item Posredna reprezentacija visokog nivoa
    \item Tipizirana reprezentacija visokog nivoa
    \item Reprezentacija srednjeg nivoa
    \item LLVM reprezentacija
\end{enumerate}


\subsection{Tok tokena}

Rust kompajler poseduje dva leksera \verb|rustc_lexer| i \verb|rustc_parse::lexer|. Oba su bitna 
jer \verb|rusc_parse::lexer| koristi \verb|rustc_lexer| tokom kompajliranja.

\subsubsection{rustc\_lexer}

\verb|rustc_lexer| je lekser niskog nivoa koji poseduje sve osnovne
funkcionalnosti potrebne prilikom prikupljanja leksema.
Glavna funkcija iz ovog modula jeste \verb|tokenize| koja na osnovu celokupnog teksta 
izvornog koda dobavlja skup tokena tj. leksema.

\begin{verbatim}
/// Creates an iterator that produces tokens from the input string.
pub fn tokenize(input: &str) -> impl Iterator<Item = Token> + '_ {
    let mut cursor = Cursor::new(input);
    std::iter::from_fn(move || {
        let token = cursor.advance_token();
        if token.kind != TokenKind::Eof { Some(token) } else { None }
    })
}
\end{verbatim}

Implementacija je bazirana na kursoru i unutar te strukture se nalazi celokupna logika leksera. 
Kursor je realizovan pomoću Rust-ovog iteratora i na osnovu njega prati trenutnu poziciju 
unutar izvornog koda. Veoma je bitna mogućnost gledanja ispred (\verb|look-ahead|) koja omogućuje 
izvršavanje u jednom prolasku.

\begin{verbatim}
    /// Peeks the first symbol from the input stream without consuming it.
    pub fn first(&self) -> char {
        // `.next()` optimizes better than `.nth(0)`
        self.chars.clone().next().unwrap_or(EOF_CHAR)
    }

    /// Peeks the second symbol from the input stream without consuming it.
    pub(crate) fn second(&self) -> char {
        // `.next()` optimizes better than `.nth(1)`
        let mut iter = self.chars.clone();
        iter.next();
        iter.next().unwrap_or(EOF_CHAR)
    }

    /// Peeks the third symbol from the input stream without consuming it.
    pub fn third(&self) -> char {
        // `.next()` optimizes better than `.nth(1)`
        let mut iter = self.chars.clone();
        iter.next();
        iter.next();
        iter.next().unwrap_or(EOF_CHAR)
    }
\end{verbatim}
Kloniranje iteratora je jeftina operacija jer se svodi na kopiranje meta podataka
kao što je trenutna adresa.

\verb|Token| sadrži samo informaciju o tipu token-a i njegovu dužinu ali ne i 
sam podatak.

\begin{verbatim}
#[derive(Debug)]
pub struct Token {
    pub kind: TokenKind,
    pub len: u32,
}   
\end{verbatim}

\subsubsection{rustc\_parse::lexer}

Lekser višeg nivoa \verb|rustc_parse::lexer| koristi \verb|rustc_lexer| prilikom izvršavanja sopstvenih
operacija. Bitna razlika je što se sadržaj token analizira i postavlja u kontekst.
Lekser višeg nivoa koristi kursor svog prethodnika za dobavljanje leksema ali unutar strukture \verb|StringReader|. 

\begin{verbatim}
    struct StringReader<'psess, 'src> {
        psess: &'psess ParseSess,
        start_pos: BytePos,
        pos: BytePos,
        src: &'src str,
        cursor: Cursor<'src>,
        override_span: Option<Span>,
        nbsp_is_whitespace: bool,
        last_lifetime: Option<Span>,
    }
\end{verbatim}
\verb|StringReader| prati trenutnu poziciju unutar izvornog koda, a kursor 
dobavlja token koji ima odredjenu dužinu. Sadržaj tokena je niz karaktera sa početkom u \verb|pos|,
dužine od \verb|token.len|.

Prolaskom kroz izvorni kod izvršavaju se sledeće akcije:
\begin{enumerate}
    \item Interniranje 
    \item Tokeni iz \verb|rustc_lexer| se pretvaraju u kompleksnije tokene iz \verb|rustc_ast|
    \item Rezulucija zagrada svih tipova.
    \item Problemi i preporuke generišu dijagnostiku (deo parsiranja)
\end{enumerate}

Interniranje je proces u kome se stringovi pretvaraju internirane simbole. Na ovaj način
se čuva samo jedna nepromenljiva kopija sadržaja tokena. Tabela simbola je struktura podataka
koja čuva i održava O(1) pristup bilo kom simbolu.  Tabela simbola je implementirana pomoću \verb|IndexSet|
strukture. Ne internira se svaki token, već samo tokeni koji imaju varijabilnu dužinu.


\section{Reference}
[1] MSRC, “A proactive approach to more secure code | MSRC Blog | 
Microsoft Security Response Center,” Microsoft.com, Jul. 16, 2019. 
\url{https://msrc.microsoft.com/blog/2019/07/a-proactive-approach-to-more-secure-code/} 
(pristupljeno Sep. 08, 2024).

\end{document}
