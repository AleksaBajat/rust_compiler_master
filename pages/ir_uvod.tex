\section{Medju reprezentacije izvornog koda}

Prilikom kompajliranja \verb|Rust| izvornog koda, \verb|Rust| \verb|frontend| mora da osigura da je 
kod sintaktički i semantički tačan da bi mogao da izgeneriše pravilnu \verb|LLVM| medju reprezentaciju 
na osnovu koje se generiše mašinski kod. Da bi se ovo efektivno izvelo \verb|Rust| \verb|frontend| poseduje 
sopstvene medju reprezentacije koje obavljaju distinktne dužnosti. 

Leksički analizator ili lekser je ulazna tačka kompajlera. Lekser čita izvorni kod kao tok karaktera i grupiše karaktere 
u sekvence koje imaju značenje. Ovaj proces se naziva leksička analiza ili skeniranje, a sekvence karaktera se nazivaju lekseme. 
Za svaku leksemu leksički analizator proizvodi token. Tokeni se obično sastoje od tipa i vrednosti.
Leksički analizator je obično najjednostavniji deo kompajlera za implementirati. 
Uobičajno je da se lekseri programskih jezika implementiraju kao konačni automat uz pomoć alata kao što 
je \verb|Flex|. Ipak, u \verb|Rust| jeziku lekser je implementiran ručno, što omogućava finu granularnost 
i visoku kontrolu nad procesom tokenizacije. Imajući u vidu da je jedan od ciljeva \verb|Rust| jezika 
da poseduje veoma detaljne i korisne poruke o grešci, ovo je vrlo promišljena implementaciona odluka. 
Tok tokena nije naročito korisna forma za analizu.

Naredni korak u procesu kompaljiranja jeste parser koji vrši sintaksnu analizu ili parsiranje. 
Parser koristi tokene koje je izgenerisao lekser da bi kreirao strukturu oblika stabla koja se naziva 
apstraktno sintaksno stablo. Apstraktno sintaksno stablo eksplicitno naznačava prvenstvo operatora,
i relativno ga je lako validirati.

Poslednji korak u \verb|Rust| \verb|frontend|-u jeste semantička analiza. Semantička analiza se sprovodi 
kroz tri medju reprezentacije, reprezentacija visokog nivoa, tipizirana reprezentacija visokog nivoa
i reprezentacija srednjeg nivoa. Bitan deo semantičke analize jeste provera tipova gde kompajler proverava 
da li svaki operator ima operande koji se poklapaju. U drugim jezicima često je postojanje implicitnih 
konverzija jednog tipa u drugi, ova funkcionalnost nije postojana u \verb|Rust| jeziku. Svaka konverzija 
mora biti eksplicitno navedena od strane korisnika.

U tipično struktuiranom kompajleru svaka od prethodno navedenih medju reprezentacija koda predstavlja 
poseban program i zasebno izvršavanje. Ovakvi kompajleri se nazivaju 
kompajleri zasnovani na prolascima. \verb|Rust|-ov kompajler je u tranziciji izmedju 
kompajlera zanovanog na prolascima i kompajlera zasnovanog na potražnji. Kompajler 
zasnovan na potražnji umesto da radi na principu arhitekture sa cevima (\verb|pipeline|)
koristi upite (\verb|query|) nad izvornim kodom i simultono izvršava celokupan proces.

Izvršavanje upita je memoizovano. Prvi put kada se upit pozove, izvršiće se komputacije 
vezane za taj upit. Naredni poziv istog upita rezultat dobavlja iz rečnika.
Ovakav princip izvršavanja se savršeno uklapa u okvir inkrementalne kompilacije. 
Rezultat prethodne kompilacije se čuva u masovnoj memoriji (skladištu) koji se može iskoristi
u trenutnoj kompilaciji.

Trenutno se svaka reprezentacija izvornog koda na \verb|frontend|-u, 
nakon kreiranja apstraktnog sintaksnog stabla, zasniva na upitima. 
Lekser i parser još uvek funkcionišu na osnovu prolazaka, 
ali cilj je da se i oni refaktorišu kako bi koristili upite.

Ulazna tačka kompajlera u \verb|rustc| izvornom kodu se nalazi u \verb|rustc_driver_impl| \verb|crate|-u
koji poseduje funkciju \verb|run_compiler|. Ova funkcija daje veoma dobar pregled toka kompajliranja 
i značajna je tačka prlikom upoznavanja sa izvornim kodom. 