\section{Rust kompajler}

Rust kompajler u svakom trenutku poseduje tri glavne verzije skupa alata (\verb|toolchain|): stabilna, beta i noćna (\verb|nightly|) verzija.
Stabilna verzija kompajlera je bezbedna i preporučena za produkcionu upotrebu. Beta verzija sadrži funkcionalnosti
koje se usvajaju u narednoj stabilnoj verziji. Noćna verzija je verzija kompajlera koja se 
proizvodi svaku noć, sa svim najnovijim netestiranim promenama. Kapija funkcionalnosti je mehanizam bezbednosti
koji onemogućava da eksperimentalne (nestabilne) funkcionalnosti slučajno završe u produkciji. 
Kapija je definisana na osnovu stanja kapije, naziva funkcionalnosti, verzije, broja zahteva za povlačenjem
(\verb|pull| \verb|request|) i opcionog komentara.
Stanje kapije funkcionalnosti može biti nestabilno, obrisano ili prihvaćeno i nekompletirano. Nekompletirano 
stanje unutar kapije daje do znanja korisniku noćne verzije ili drugim saradnicima da je nestabilna funkcionalnost 
u razvoju.  Nestabilne kapije onemogućavaju da stabilna i beta verzija kompajlera kompajliraju 
dati segment koda (ignoriše se).  Samim time razvoj je direktno 
usmeren ka noćnoj verziji koja je pod nadzorom celokupnog Rust tima. Obrisane kapije se ne brišu fizički iz koda,
već se status kapije promeni u obrisan uz komentar sa razlogom. Na ovaj način se čuva istorija pokušanih 
funkcionalnosti sa razlozima koji su uzrokovali brisanje. U slučaju da se sličan zahtev ponovo javi, retrospektiva
služi kao moćan alat koji će idealno upozoriti saradnike i preusmeravati ih na bolji pravac.

\begin{listing}[H]
\begin{minted}{rust}
    (incomplete, pub_restricted, "CURRENT_RUSTC_VERSION", Some(32409)),
    (unstable, pub_restricted, "CURRENT_RUSTC_VERSION", Some(32409))
    (accepted, pub_restricted, "CURRENT_RUSTC_VERSION", Some(32409)),
    (removed, pub_restricted, "CURRENT_RUSTC_VERSION", 
    Some(32409), Some("Removed because.."))
\end{minted}
\caption{Kapija funkcionalnosti}
\label{lst:rustup_set}
\end{listing}

Svaka funkcionalnost prolazi kroz dugačak proces provera pre nego 
što se usvoji. Ukoliko funkcionalnost zahteva promene koje nisu reorganizacija, refaktorisanje, dokumentovanje
i slično, kreira se zahtev za komentare (\verb|RFC|) koji detaljno objašnjava važnost funkcionalnosti i 
pregled implementacionih detalja sa visine. U suprotnom ovaj segment se preskače. Stejkholderi kao što je Rust kompajler tim, ali i sama Rust zajednica 
ima pravo da vaga o potencijalnim implikacijama koje funkcionalnost donosi. Odobravanje zahteva znači početak 
razvoja ali ne garantuje usvajanje. Ukoliko je funkcionalnost razvijena, pokrivena testovima, 
bez značajnih primedbi, bilo koji saradnik može pokrenuti zahtev za stabilizaciju. Zahtev je konstituiran iz 
četiri dela: dokumentacionog \verb|pull| \verb|request|-a, stabilizacionog izveštaja, 
perioda finalnog komentara i stabilizacionog \verb|pull| \verb|request|-a.

Dokumentacija funkcionalnosti koja se stabilizuje se briše iz nestabilne knjige \cite{unstable} i ažurira se 
dokumentacija namenjena korisnicima jezika. Korisnici povlače informacije iz različitih delova dokumentacije,
naime najbitnije je detljno ažurirati knjigu referenci \cite{rust-reference} koja opisuje svaku stabilnu 
karakteristiku jezika. Nestablina knjiga vodi relativno ažurnu evidenciju o nestabilnim funkcionalnostima.
Svaka nestabilna funkcionalnost pripada samo jednoj od tri potkategorije: zastavica kompajlera, funkcionalnost jezika ili
funkcionalnost standardne biblioteke. Stabilizacioni izveštaj sadrži primere koji prikazuju novu karakteristiku 
u realnom scenariju, linkove ka dokumentaciji, kao i objašnjenje podržano testovima koji opisuju ponašanje funkcionalnosti u 
graničnim slučajevima. Saradnik koji prati tok razvoja funkcionalnosti, ukoliko se slaže sa procesom 
stabilizacije, započinje period finalnog komentara. Ostatak tima pregleda predlog i ukoliko je koncenzus
pozitivan finalni zahtev za povlačenjem se kreira. Cilj je da se promeni status zastavice u prihvaćeno
stanje kao i brisanje makroa \verb|gate_feature_post!| koji pokreće grešku ukoliko se funkcionalnost koristi van noćne verzije kompajlera.
Ukoliko je sve zadovoljeno, funkcionalnost postaje bliža krajnjim korisnicima za testiranje u beta verziji, dok
neminovno ne predje u stabilnu verziju u narednoj stabilnoj distribuciji. 

\newpage

Originalna ideja iza nestabilnih funkcionalnosti kompajlera, pored očiglednih bezbednosnih pogodnosti, 
jeste da se pruži pristup najnovijim karakteristikama zarad testiranja i evaluacije korisnosti.
Uprkos tome 2022. godine 12\%  paketa u Rust ekosistemu se direktno oslanjalo na nestabilne funkcionalnosti, a 
44\% paketa je indirektno zavisilo od nestabilnih funkcionalnosti da bi se kompajliralo \cite{unstable-flags}. Iako 
statusi kapija semantički omogućavaju brisanje kapije bez grubog brisanja, kompajler konstantno menja arhitekturu, čime ujedno 
rešava probleme zbog kojih je postojanje nestabilnih funkcionalnosti bilo potrebno. Naime brisanje kapije 
u bilo kom obliku označava da u sledećoj noćnoj verziji, program ili biblioteka zavisna od nestabilne funkcionalnosti
neće moći da se kompajlira. To takodje ne garantuje da se nestabilna funkcionalnosti neće vratiti već u sledećoj
verziji i obrnuto. Iako se od strane zajednice pruža trud da se nestabilne funkcionalnosti prate, ovakav vid 
razvoja bi se pokazao neefikasan jer svaki budući razvoj potencijalno zahteva obazrivost na nestabilne 
biblioteke što nije cilj kompajlera. Procentualna zavisnost prema nestabilnim funkcionalnostima je visoka ali 
analiza nije sprovedena da se izračuna procenat masovno korišćenih paketa sa direktnim ili tranzitivnim
zavisnostima, kao i procenat nestabilnih funkcionalnosti na kojima se ovakvi paketi zasnivaju.

Skup alata (\verb|toolchain|) je kompletna instalacija Rust kompajlera (\verb|rustc|) i srodnih alata kao što 
je \verb|Cargo|. Skup alata se instalira i ažurira uz pomoć programa komandne linije \verb|rustup| \ref{lst:rustup_set}. 

\begin{listing}[H]
\begin{minted}{text}
    rustup toolchain install [version]
\end{minted}
\caption{Instaliranje novog skupa alata}
\label{lst:rustup_set}
\end{listing}

Program \verb|rustup| automatski odredjuje koji skup alata će se koristiti prilikom pokretanja kompajlera. 
Postoji više načina da se kontroliše koji skup alata se primenjuje za neko proizvoljno okruženje.
Ako prvi argument \verb|rustc| kompajlera ili \verb|Cargo| menadžera paketa počinje sa "+" (bez navodnika) 
onda naznačava ime skupa alata koji je poželjan da se koristi \ref{lst:toolchain_cli}. 

\begin{listing}[H]
\begin{minted}{text}
    cargo +beta run 
    rustc +beta main.rs
\end{minted}
\caption{Konfigurisanje skupa alata kroz argumente komandne linije}
\label{lst:toolchain_cli}
\end{listing}

Ukoliko ne postoji prikazuje se poruka da skup 
alata nije instaliran ili da ga je nemoguće dobaviti. Promenljiva okruženja \verb|RUSTUP_TOOLCHAIN| specificira 
podrazumevani skup alata. Skup alata je moguće primeniti na nivou direktorijuma upotrebom \verb|rustup| \verb|override|
komande \ref{lst:toolchain_dir}. 

\begin{listing}[H]
\begin{minted}{text}
    rustup override set beta 
\end{minted}
\caption{Konfigurisanje skupa alata na nivou direktorijuma}
\label{lst:toolchain_dir}
\end{listing}

Alternativno kreiranjem konfiguracionog fajla \verb|rust-toolchain.toml| u korenskom 
direktorijumu projekta moguće je specificirati skup alata za taj projekat \ref{lst:toolchain_cfg}.

\begin{listing}[H]
\begin{minted}{toml}
[toolchain]
channel = "nightly-2020-07-10"
components = [ "rustfmt", "rustc-dev" ]
targets = [ "wasm32-unknown-unknown", "thumbv2-none-eabi" ]
profile = "minimal"
\end{minted}
\caption{Konfigurisanje skupa alata uz pomoć konfiguracionog fajla}
\label{lst:toolchain_cfg}
\end{listing}

Ovaj način je najoptimalniji u produkcionim projektima ukoliko se koristi skup alata koji 
nije podrazumevan budući da je konfiguracioni fajl daleko opsežniji i obuhvata mete kompajliranja i 
potrebne komponente. Mete kompajliranja su distribucije za koje se kompaljira \verb|Rust| program.
Svaki skup alata sadrži komponente od kojih su neke opcione a neke neophodne. Jedna od najkorišćenijih 
opcionih komponenti jeste \verb|clippy| koji služi da predupredi što više čestih grešaka uz pomoć 
700 novih upozorenja. 


Za prikaz dostupnih lokalnih verzija, kao i trenutna aktivna verzija u direktorijumu koristi se \verb|rustup|
\verb|show| komanda \ref{lst:rustup_show}.

\begin{listing}[H]
\begin{minted}{text}
    rustup show
        installed toolchains
        --------------------
        stable-x86_64-unknown-linux-gnu (default)
        nightly-x86_64-unknown-linux-gnu
        active toolchain
        ----------------
        nightly-x86_64-unknown-linux-gnu (directory 
        override for '/home/abajat/Documents/projects/master')
        rustc 1.83.0-nightly (12b26c13f 2024-09-07)
\end{minted}
\caption{Prikaz izlaza "rustup show" komande}
\label{lst:rustup_show}
\end{listing}

Ključna osobina \verb|Rust| jezika jeste stabilnost bez stagnacije. Jednom kada se nova 
funkcionalnost jezika objavi u stabilnoj verziji, kontributori će održavati 
tu funkcionalnost u svim narednim verzijama \cite{editions}.
Rane verzije Rust-a nisu posedovale \verb|async| i \verb|await| ključne reči.
Kada bi Rust bez mera predostrožnosti uveo nove ključne reči, neka količina koda bi bila pokvarena.

\begin{listing}[h]
\begin{minted}{rust}
    // validan rust kod 
    // ali nije validan ako je edicija posle 2018 uključena
    let async = 5; 
\end{minted}
\caption{Nekompatibinost prilikom promene edicije}
\label{lst:edition}
\end{listing}


Rust koristi edicije da bi rešio ovaj problem. Ako postoje promene koje su unazad 
nekompatibilne onda postaju sastav sledeće edicije. Edicija se mora eksplicitno navesti 
u \verb|Cargo.toml| konfiguracionom fajlu da bi se funkcionalnosti uključile \ref{lst:edition_toml}. Kreiranje novog projekta uz 
pomoć menadžera paketa \verb|Cargo| automatski bira poslednju dostupnu verziju.
\verb|Crate|-ovi različitih edicija su kompatibilne izmedju sebe. Kompatibilnost omogućuje 
karakteristika Rust kompajlera da prilikom kompajliranja prevodi svaki \verb|Crate| na istu medju reprezentaciju 
koda bez obzira na ediciju. Omogućavanje ovakvog nivoa kompatibilnosti limitira nivo promene 
koji se može očekivati u ediciji. Migracija na novu verziju \verb|Rust|-a je obično laka i pretežno automatizovana
i sprovodi se kroz \verb|Cargo|. Automatske migracije nisu savršene i moguće je da je potreban nivo manuelnog 
truda da bi se u potpunosti završio proces.

\begin{listing}[h]
\begin{minted}{toml}
[package]
name = "some_crate"
version = "0.1.0"
edition = "2021"

\end{minted}
\caption{Eksplicitno navodjenje edicije u Cargo.toml fajlu}
\label{lst:edition_toml}
\end{listing}

\newpage


Podrazumevano ponašanje \verb|rustc| kompajlera moguće je promeniti upotrebom promenljivih okruženja 
i zastavica funkcionalnosti (\verb|feature| \verb|flags|). Kompletan spisak omogućenih zastavica funkcionalnosti 
za korišćeni skup alata se može prikazati upotrebom \verb|--help| opcije \verb|rustc| kompajlera \ref{lst:rustc_flags}.

\begin{listing}[H]
\begin{minted}{bash}
    rustc --help -v 
\end{minted}
\caption{Prikaz svih omogućenih zastavica funkcionalnosti}
\label{lst:rustc_flags}
\end{listing}

Glavni podskup zastavica funkcionalnosti se bavi generisanjem koda. Navode se korišćenjem \verb|-C| prefiksa. 
Budući da generisanje koda izvršava 
LLVM set alata, dostupan je širok dijapazon opcija. Nivoi optimizacije koda, kao i ciljani procesor za koji 
se kod kompajlira su neka od bitnijih podešavanja. Zastavica \verb|emit| je izuzetno bitna jer omogućava izbor 
dodatnih izlaznih artefakta koji se traže od kompajlera. Bitniji artefakti su LLVM bitkod, LLVM medju 
reprezentacija i asemblerski kod.

\begin{listing}[H]
\begin{minted}{bash}
    cargo rustc -- --emit=llvm-bc
    cargo rustc -- --emit=llvm-ir
    cargo rustc -- --emit=asm
\end{minted}
\caption{Generisanje dodatnih izlaznih artefakta}
\label{lst:emit_flag}
\end{listing}


\verb|Rust| se odlikuje sa dva načina kompilacije, kompletna 
i inkrementalna. Kompletna kompilacija je podrazumevana i \verb|Crate| se kompajlira od početka svaki put.
Inkrementalna kompilacija se uključuje uz pomoć zastavica funkcionalnosti gde se prilikom kompajliranja 
beleže izlazi koji mogu da se iskoriste u narednom procesu kompajliranja \ref{lst:incremental_flag}. Inkrementalna kompilacija je 
efektivna u slučaju da je \verb|Crate| veoma velik i smanjuje vreme sekvencijalnih kompilacija. Prilikom ovog 
vida kompilacije čuva se značajno više podataka tj. dodatne performanse se dobijaju na uštrb memorije na disku.
Samim time ovo nije efektivan metod kompajliranja u \verb|CI/CD| okruženjima jer agent koji vrši kompajliranje
obično nije deterministički. Keširanje izlaza inkrementalne kompilacije i korišćenje istog u nesekvncijalnoj 
kompilaciji može biti sporije nego kompletno kompajliranje usled bačenog procesorskog vremena na proveru 
izlaza koji se ne mogu ponovo iskoristiti.

\begin{listing}[H]
\begin{minted}{text}
    cargo rustc -- -C incremental=true 
\end{minted}
\caption{Inkrementalna kompilacija Crate-a}
\label{lst:incremental_flag}
\end{listing}


Noćna verzija \verb|Rust| kompajlera se odlikuje proširenim skupom zastavica funkcionalnosti koje se zovu 
nestabline zastavice funkcionalnosti. Navode se korišćenjem \verb|-Z| prefiksa. U ovom skupu se nalaze 
zastavice koje su izuzetno korisne prilikom razvoja \verb|rustc|-a ali i zastavice koje se razvijaju
i koje će potencijalno biti deo nove verzije \verb|rustc|-a. Jedna od najbitnijih zastavica u razvoju 
jeste \verb|unpretty| koja može da prikaže izvorni kod u nekoj medju reprezentaciji kao što su
apstraktno sintaksno stablo, visoka medju reprezentacija i srednja medju reprezetacija \ref{lst:intermediate_representation}.

\begin{listing}[H]
\begin{minted}{bash}
    # Apstraktno sintaksno stablo 
    cargo +nightly rustc -- -Z unpretty=ast-tree
    # Visoka medju reprezentacija
    cargo +nightly rustc -- -Z unpretty=hir
    # Srednja medju reprezentacija
    cargo +nightly rustc -- -Z unpretty=mir
\end{minted}
\caption{Prikaz medju reprezentacija izvornog koda}
\label{lst:intermediate_representation}
\end{listing}