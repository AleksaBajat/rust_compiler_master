\subsection{Apstraktno Sintaksno Stablo}

Parser jezika Rust uz pomoć toka tokena iz leksera generiše apstraktno sintaksno stablo (AST). 
Prilikom poziva parsiranja, jedini fajl koji se odmah procesira jeste glavni \verb|main| fajl, potpuno stablo
nastaje putem ekspanzije i rezolucije imena.
Algoritam \verb|recursive-descent| se koristi tokom procesa parsiranja. Ovo je jedna od najjednostavnijih
tehnika parsiranja koja se koristi u praksi. Ovakvi parseri se takodje nazivaju \verb|top-down| parseri 
jer konstruišu stablo od gore ka dole \cite{parsing}. 

Parser koristi prethodno pomenuti tok tokena da bi kreirao kursor nad stablom tokena \verb|TokenTreeCursor|.
\verb|TokenStream| poseduje implementaciju \verb|into_tree| koji transformiše tip dobijen iz leksera
u tip pogodan parseru.

\begin{listing}[H]
\begin{minted}{rust}
pub fn into_trees(self) -> TokenTreeCursor {
    TokenTreeCursor::new(self)
}

impl TokenTreeCursor {
    fn new(stream: TokenStream) -> Self {
        TokenTreeCursor { stream, index: 0 }
    }
    ...
}
\end{minted}
\caption{Konverzija iz "TokenStream" u "TokenTreeCursor"}
\end{listing}
Jednostavnije je iterirati kroz stabla tokena sekvencijalno u odnosu na strukturu koja liči na kanape.

Za razliku od leksera, parser poseduje konstrukte koji su dosta složeniji od jednog tokena, kao što su 
strukture, enumeracije i slično (reč-rečenica poredjenje). Izvorni čvor \verb|AST|-a je \verb|Crate|.

\begin{listing}[H]
\begin{minted}{rust}
#[derive(Clone, Encodable, Decodable, Debug)]
pub struct Crate {
    pub attrs: AttrVec,
    pub items: ThinVec<P<Item>>,
    pub spans: ModSpans,
    pub id: NodeId,
    pub is_placeholder: bool,
}
\end{minted}
\caption{Definicija "Crate" strukture}
\end{listing}

\newpage

Najbitnija polja su atributi (\verb|attrs|) i stavke (\verb|items|). Atributi mogu biti spoljašnji ili unutrašnji.
Koriste se da bi korisnik pružio dodatne instrukcije kompajleru. 
Dodatne instrukcije mogu biti trivijalne poput dozvoljavanja
nekorišćenog koda, ali mogu biti i kompleksnije u vidu implementacije osobina 
(\verb|trait|). Osobina je funkcionalnost veoma slična interfejsu u drugim jezicima. 




Stavke su grupacije tokena iz leksera koje tako grupisane imaju semantiku. Iz perspektive koda 
veoma se malo razlikuje od samog \verb|Crate|-a. Svaka stavka poseduje identifikator \verb|NodeId| 
koji predstavlja sekvencijalni broj stavke unutar \verb|Crate|-a. Ovo znači da ukoliko se usred stabla 
obriše ili izgeneriše nova stavka, svaki čvor nakon tog dela stabla mora ponovo da izgeneriše svoj redni broj.
Samim time apstraktno sintaksno stablo nije korisno prilikom inkrementalne kompilacije jer su identifikatori
volotilni. Svaka stavka takodje sadrži \verb|Span|. Ovo je tip koji odredjuje početnu i krajnju poziciju 
stavke unutar izvornog koda. U odnosu na vrednost tipa \verb|kind|, stavke mogu ili ne mogu da poseduju atribute. 

\begin{listing}[H]
\begin{minted}{rust}
#[derive(Clone, Encodable, Decodable, Debug)]
pub struct Item<K = ItemKind> {
    pub attrs: AttrVec,
    pub id: NodeId,
    pub span: Span,
    pub vis: Visibility,
    pub ident: Ident,
    pub kind: K,
    pub tokens: Option<LazyAttrTokenStream>,
}
\end{minted}
\caption{Definicija "Item" strukture}
\end{listing}

\begin{listing}[H]
\begin{minted}{rust}
pub fn parse_crate_mod(&mut self) -> PResult<'a, ast::Crate> {
    let (attrs, items, spans) = self.parse_mod(&token::Eof)?;
    Ok(ast::Crate { attrs, items, spans, 
       id: DUMMY_NODE_ID, is_placeholder: false })
}
\end{minted}
\caption{Ulazna funkcija parsera}
\end{listing}


Rezolucija imena je jedan od delova formiranja apstraktnog sintaksnog stabla.
U programima se referenciraju funkcije, varijable i tipovi na osnovu imena. Ova imena nisu uvek jedinstvena.

\begin{listing}[H]
\begin{minted}{rust}
type a = u32;
let a: a = 1;
let b: a = 5; 
\end{minted}
\caption{Rezolucija imena}
\label{lst:resolution}
\end{listing}

Kako znamo, u izvornom kodu \ref{lst:resolution}, da li je u liniji 3 "a" tip ili vrednost 1?
Ovakvi konflikti se rešavaju prilikom rezolucije imena. U datom slučaju imena tipova i promenljivih se nalaze 
u različitim imenskim prostorima (\verb|namespaces|) i zbog toga mogu da koegzistiraju.
Rezolucija imena se izvodi iz dva faze. Prva faza se dešava za vreme ekspanzije (više o tome u daljem tekstu) 
gde se obradjuju samo \verb|import|-i. Druga faza se dešava nakon ekspanzije gde se obradjuje celokupno 
sintaksno stablo počevši od vrha pa na dole. 

Imena su validna u različitim delovima izvornog koda. Da bi se 
vreme života (\verb|scope|) nekog imena proverilo uvodi se koncept rebara (\verb|rib|). Svaki put kada se 
set vidljivih imena potencijalno menja novo rebro se gura na stek. Primeri situacija u kojima se dodaje 
novo rebro:
\begin{enumerate}
    \item Očigledna mesta kao što su vitičaste zagrade, funkcije, moduli.
    \item Prilikom poziva \verb|let| jer se ime može redefinisati (\verb|shadowing|).
    \item Početak proširenja makroa. 
\end{enumerate}
Potraga za imenom se izvodi od najdubljeg rebra (vrh steka) ka najplićem. 
Redosled je veoma bitan zato što se ovakvom pretragom izbegava greška 
ukoliko je neko ime redefinisano.

\begin{listing}[H]
\begin{minted}{rust}
let a: u32 = 1;
let a: i32 = -1;
\end{minted}
\caption{"Shadowing"}
\label{lst:shadowing}
\end{listing}

Jezici kao što je C pored kompajlera sadrže pre-procesor koji prikuplja izvorni kod. 
Prilikom prikupljanja izvornog koda skraćenice (makroi) se proširuju na izraze programskog jezika. 
Rust se odlikuje veoma moćnim makroima ali ne poseduje pre-procesor. Transformacija skraćenica se vrši 
putem ekspanzije.  Ekspanzija je proces tokom kog se dopunjuje apstraktno sintaksno stablo
proširivnajem makroa i umetanjem modula (\verb|inlining|). Prilikom parsiranja umesto 
modula i makroa se postavljaju AST fragmenti. Ovi fragmenti su početne tačke algoritma ekspanzije.
Kreira se red čekanja neproširenih makroa. Jedan po jedan se uzimaju iz reda čekanja, vriši se rezolucija imena,
proširuju se i integrišu nazad u stablo. U slučaju da se nema napretka (loše napisan kod), vraća se greška.

Pre nego što počne prelazak u sledeći posredni oblik AST stablo se validira. 
Tokom prolaska se proverava da li je stablo sintaktički validno, bez ikakvih kompleksnih analiza,
rezolucije imena ili provera tipova. Ovo nije moguće učiniti prilikom parsiranja jer makro atributi 
omogućavaju odredjene delove netačne sintakse kao što je funkcija bez tela van definicije osobine (\verb|trait|) \ref{lst:validate}.

\begin{listing}[H]
\begin{minted}{rust}
#[proc_macro_attribute]
pub fn expand_body(_attr: TokenStream, _item: TokenStream) -> TokenStream {
    let fun = format!("fn missing_body() 
    {{ println!(\"I'm not missing body anymore.\"); }}");
    let stream: TokenStream = fun.parse().unwrap();
    stream
}
#[expand_body]
fn missing_body();
fn main() {
    missing_body();
}
\end{minted}
\caption{Dodavanje tela funkcije uz pomoć makroa}
\label{lst:validate}
\end{listing}

Apstraktno sintaksno stablo pre i posle ekspanzije nekog \verb|Crate|-a se može analizirati pomoću 
nestabilnih \verb|-Z| zastavica \ref{lst:rustc_ast}.

\begin{listing}[H]
\begin{minted}{bash}
cargo rustc -- -Z unpretty=ast
cargo rustc -- -Z unpretty=ast,expanded
\end{minted}
\caption{"Prikaz apstraktnog sintaksnog stabla"}
\label{lst:rustc_ast}
\end{listing}