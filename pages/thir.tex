\subsection{Tipizirana medjureprezentacija visokog nivoa - THIR (Typed HIR)}

Tipizirana medjureprezentacija visokog nivoa je medjureprezentacija izvornog koda koja nastaje dopunom stabla tipovima.
Za razliku od \verb|HIR| medjureprezentacije, \verb|THIR| sadrži samo tela tj. izvršni kod. To znači da \verb|THIR| 
ne sadrži reprezentaciju stavki kao što su strukture (\verb|struct|) i osobine (\verb|traits|). Svako telo u ovoj reprezentaciji
se čuva priveremeno u memoriji i odbačeno čim više nije potrebno. Ovo je bitna distinkcija u donosu na \verb|HIR| gde se reprezentacija 
čuva tokom celokupnog procesa kompilacije. Dodatno, automatska referenciranja i dereferenciranja su eksplicitna, pozivi metoda i 
preklopljeni operatori su pretvoreni u obične pozive funkcija. Uništenje opsega je u ovoj reprezentaciji eksplicitno.
Izrazi, iskazi i klauzule \verb|match| odredbe se čuvaju posebno.

\subsubsection{Bezbednost}

Tipizirana medjureprezentacija ima značajnu ulogu u proveri bezbednosti koda. Provere bezbednosti se nalaze u modulu \verb|check_unsafety|.
Algoritam prolazi kroz telo funkcije i sve njene anonimne funkcije prateći da li je nebezbednom \verb|unsafe| kontekstu.
Ukoliko se nebezbedan kod poziva van nebezbednog (\verb|unsafe|) bloka greška će biti prikazana. Algoirtam takodje vodi računa da li 
postoji nebezbedan blok u kome se ne koristi nebezbedan kod. Ako postoji ovakav blok kompajler će prikazati upozorenje (\verb|lint|).

Dodatak \ref{lst:safety_check} prikazuje isečak glavne funkcije koja je odgovorna za opseg bezbednosti. 
Na osnovu \verb|LocalDefId| dobavlja se telo u \verb|THIR| medjureprezentaciji. Potom se na osnovu istog 
identifikatora evaluira \verb|HirId| uz pomoć kojeg se vrši upit nad \verb|HIR|-om čime se otkriva početni kontekst bezbednosti.
Struktura \verb|UnsafeVisitor| koristi \verb|THIR| telo, na osnovu kog sprovodi prethodno objašnjen algoritam. Može se primetiti 
da se ovde incijalizuje vektor upozorenja koja se prikazuju ukoliko se nebezbedan kod ne koristi u nebezbednom kontekstu.

\subsubsection{Provera iscrpnosti}
Rust ima odliku da ukoliko se koristi neki vid provere šablona (\verb|pattern matching|) svaki tip unutar 
šablona mora biti pokriven \ref{lst:pattern_matching}. Proveru da li je svaki tip obradjen izvršava THIR. 

\begin{listing}[H]
\begin{minted}{rust}
pub enum IpAddrKind {
    V4,
    V6,
}

fn main() {
    let x = IpAddrKind::V4;
    match x {
        IpAddrKind::V4 => println!("Ovo je IPv4 adresa."), 
        // Pokriven IpAddrKind::V4
        IpAddrKind::V6 => println!("Ovo je IPv6 adresa.") 
        // Pokriven IpAddrKind::V6
        // Da je postojao IpAddrKind::V7 Rust bi zahtevao 
        // da i ta varijanta bude obradjena.
    }
}
\end{minted}
\caption{Provera šablona}
\label{lst:pattern_matching}
\end{listing}

Za razliku od HIR-a koji je perzistiran tokom celokupnog izvršavanja kompajlera, THIR se odbacuje momenta
kada više nema upotrebnu vrednost. Pored toga što su svi tipovi čvorova prisutni, THIR se odlukuje dodatnim 
pojednostavljivanjem koda:
\begin{enumerate}
    \item Automatska referenciranja i dereferenciranja su eksplicitna.
    \item Pozivi metoda i opterećeni operatori su konvertovani u obične pozive funkcija. 
    \item Obim životnog veka je eksplicitan.
\end{enumerate}
% Izrazi, iskazi i \verb|match| ruke se čuvaju zasebno. 

Pojednostavljenje čini THIR podobnim za proveru nebezbednog koda jer fundamentalno poseduje manje opcija, 
nebezbedni pozivi metoda i nebezbedni pozivi funkcija imaju identičnu reprezentaciju. Ova provera 
iskazuje grešku ako je nebezbedna operacija korišćena van \verb|unsafe| opsega.

\todo{THIR je kratak sam po sebi ali moguće je detaljnije obraditi pattern matching}